\usepackage[toc,nonumberlist]{glossaries}

\setglossarystyle{altlist}


\setacronymstyle{long-short}


\makeglossaries

\newacronym{aac}{AAC}{Advanced Audio Coding, codec audio défini par MPEG-2 partie 7 et MPEG-4 partie 3}
\newacronym{arm}{ARM}{Advanced RISC Machines, type d'architecture de processeur à faible coût et à faible consommation en énergie}
\newacronym{fpga}{FPGA}{Field-programmable gate array}
\newacronym{simd}{SIMD}{Single Instruction on Multiple Data}
\newacronym{vfp}{VFP}{Vector Floating Point, type de FPU spécifique à l'architecture ARM}
\newacronym{fpu}{FPU}{Floating Point Unit, coprocesseur mathématique conçu pour réaliser des opérations mathématique en floating point}
\newacronym{pcm}{PCM}{Pulse-Code Modulation}
\newacronym{mpeg}{MPEG}{Moving Picture Experts Group}
\newacronym{mdct}{MDCT}{Modified discrete cosine transform}
\newacronym{mlt}{MLT}{Modulated Lapped Transform}
\newacronym{dct}{DCT}{Discrete cosine transform}
\newacronym{dft}{DFT}{Discrete Fourier transform}
\newacronym{fft}{FFT}{Fast Fourier transform}
\newacronym{sbc}{SBC}{single-board computer}
\newacronym{ssh}{SSH}{Secure SHELL}
\newacronym{gcc}{GCC}{GNU C Compiler}

\newglossaryentry{codec}{
    name={Codec},
    text={codec},
    description={Un codec est un procédé logiciel composé d'un encodeur (\emph{\textbf{co}der}) et d'un décodeur (\emph{\textbf{dec}oder})\cite{wiki:codec}. Un codec audio permet donc, d'une part, de coder un signal audio dans un flux de données numériques et, d'autre part, de décoder ces données afin de restituer le signal audio.}
}

\newglossaryentry{ne10}{
    name=Ne10,
    description={Librairie de fonction utiles, optimisée pour les processeurs ARM supportant les instructions NEON}
}

\newglossaryentry{transtypage}{
    name={Transtypage},
    text={transtypage},
    description={Conversion d'un type à un autre, pouvant, dans certains cas, engendrer un changement d'encodage}
}

\newglossaryentry{depassement}{
    name={Dépassement},
    text={dépassement},
    description={Condition qui se produit lorsqu'une opération mathématique produit une valeur numérique supérieure à celle représentable dans l'espace de stockage disponible. Le résultat de l'opération est alors un nombre de signe opposé à la valeur attendue}
}

\newglossaryentry{saturation}{
    name={Saturation},
    text={saturation},
    description={Lors d'une opération mathématique, si la valeur résultante sort de la limite des nombres, à la place de causer un dépassement, elle garde la valeur de la limite de son encodage}
}

\newglossaryentry{fftw3}{
    name={FFTW3},
    description={Librairie permettant de calculer des transformée de Fourier à une ou plusieurs dimensions, de taille variable en complexes ou en réel}
}

\newglossaryentry{cmake}{
    name={CMake},
    description={Système de construction logicielle multiplateforme. Il permet de vérifier les prérequis nécessaires à la construction, de déterminer les dépendances entre les différents composants d'un projet, afin de planifier une construction ordonnée et adaptée à la plateforme}
}

\newglossaryentry{arm-neon}{
    name={ARM NEON / Advanced SIMD},
    description={Technologie SIMD avancée, extension de l'architecture pour les profils A et R de processeurs}
}

\newglossaryentry{wrapper}{
    name={Wrapper},
    text={wrapper},
    description={Un conteneur (wrapper en anglais) est un fichier pouvant contenir divers types de données. Les spécifications du format conteneur décrivent la façon dont sont organisées les données à l'intérieur du fichier}
}

\newglossaryentry{intrinsic}{
    name={Intrinsèque},
    text={intrinsèque},
    description={Terme utilisé pour qualifier les fonctions implémentée directement par le compilateur, par opposition aux fonctions fournies par des libraries}
}

\newglossaryentry{raspberry}{
    name={Raspberry PI},
    description={Le Raspberry Pi est un nano-ordinateur monocarte à processeur ARM}
}

\newglossaryentry{twiddling}{
    name={Twiddling},
    text={twiddling},
    description={``Tripotage'', terme utilisé pour désigner l'arrangement nécessaire des données en entrée et sortie de fonctions vectorielles}
}

\newglossaryentry{floating-point}{
    name={Floating point},
    text={floating point},
    description={Méthode d'encodage de nombres réels utilisée par les ordinateurs, équivalente à la notation scientifique en numération binaire}
}

\newglossaryentry{fixed-point}{
    name={Fixed point},
    text={fixed point},
    description={Méthode d'encodage de nombres possédant un nombre fixe de chiffres après la virgule}
}

\glsaddallunused
