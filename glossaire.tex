\usepackage[toc,nonumberlist]{glossaries}

\setglossarystyle{altlist}

\setacronymstyle{long-short}

\makeglossaries

\newacronym{aac}{AAC}{Advanced Audio Coding, codec audio défini par les normes MPEG-2 partie 7 et MPEG-4 partie 3}
\newacronym{arm}{ARM}{Advanced RISC Machines, type d'architecture de processeur à faible coût et à faible consommation en énergie}
\newacronym{fpga}{FPGA}{Field-programmable gate array}
\newacronym{simd}{SIMD}{Single Instruction on Multiple Data}
\newacronym{vfp}{VFP}{Vector Floating Point, type de FPU spécifique à l'architecture ARM}
\newacronym{fpu}{FPU}{Floating Point Unit, coprocesseur mathématique conçu pour réaliser des opérations mathématiques en floating point}
\newacronym{pcm}{PCM}{Pulse-Code Modulation, numérisation de signal audio sans perte d'information}
\newacronym{mpeg}{MPEG}{Moving Picture Experts Group, groupe d'experts à l'origine de normes d'encodage, de décodage et de transmission de données audiovisuelles}
\newacronym{mdct}{MDCT}{Modified Discrete Cosine Transform, type de transformation utilisée pour l'analyse fréquentielle d'un signal}
\newacronym{mlt}{MLT}{Modulated Lapped Transform, fonction de fenêtre propre à la MDCT}
\newacronym{dct}{DCT}{Discrete Cosine Transform, type de transformation utilisée pour l'analyse fréquentielle d'un signal}
\newacronym{dft}{DFT}{Discrete Fourier Transform, type de transformation utilisée pour l'analyse fréquentielle d'un signal}
\newacronym{fft}{FFT}{Fast Fourier Transform, type de transformation utilisée pour l'analyse fréquentielle d'un signal}
\newacronym{sbc}{SBC}{Single-board computer, ordinateur construit sur un seul circuit}
\newacronym{ssh}{SSH}{Secure SHELL, protocole de connexion réseau}
\newacronym{gcc}{GCC}{GNU C Compiler, compilateur des langage C et C++}

\newglossaryentry{codec}{
    name={Codec},
    text={codec},
    description={Procédé logiciel composé d'un encodeur (\emph{\textbf{co}der}) et d'un décodeur (\emph{\textbf{dec}oder})}
}

\newglossaryentry{ne10}{
    name=Ne10,
    description={Librairie de fonction de traitement de signal optimisée pour les processeurs ARM supportant les instructions NEON}
}

\newglossaryentry{transtypage}{
    name={Transtypage (cast)},
    text={transtypage},
    description={Conversion d'un type à un autre, pouvant, dans certains cas, engendrer un changement d'encodage}
}

\newglossaryentry{depassement}{
    name={Dépassement},
    text={dépassement},
    description={Phénomène survenant lorsqu'une opération mathématique produit une valeur numérique supérieure à celle représentable dans l'espace de stockage disponible : le résultat de l'opération est alors un nombre de signe opposé à la valeur attendue}
}

\newglossaryentry{saturation}{
    name={Saturation},
    text={saturation},
    description={Lors d'une opération mathématique, si la valeur résultante sort de la limite des nombres représentables, au lieu de causer un dépassement, elle garde la valeur de la limite de son encodage}
}

\newglossaryentry{fftw3}{
    name={FFTW3},
    description={Librairie permettant de calculer des transformées de Fourier à une ou plusieurs dimensions, de taille variable en complexes ou en réels}
}

\newglossaryentry{cmake}{
    name={CMake},
    description={Système de construction logicielle multiplateforme. Il permet de vérifier les prérequis nécessaires à la construction et de déterminer les dépendances entre les différents composants d'un projet afin de planifier une construction ordonnée et adaptée à la plateforme}
}

\newglossaryentry{arm-neon}{
    name={ARM NEON / Advanced SIMD},
    description={Jeu d'instructions SIMD spécifique à l'architecture ARM (ARMv7 et ARMv8)}
}

\newglossaryentry{wrapper}{
    name={Wrapper},
    text={wrapper},
    description={Un conteneur (wrapper en anglais) est un fichier pouvant contenir divers types de données. Les spécifications du format conteneur décrivent la façon dont sont organisées les données à l'intérieur du fichier}
}

\newglossaryentry{intrinsic}{
    name={Fonction intrinsèque (intrinsic)},
    text={intrinsèque},
    description={Terme utilisé pour qualifier les fonctions implémentées directement par le compilateur, par opposition aux fonctions fournies par des librairies}
}

\newglossaryentry{raspberry}{
    name={Raspberry PI},
    description={Le Raspberry Pi est un nano-ordinateur monocarte à processeur ARM}
}

\newglossaryentry{twiddling}{
    name={Twiddling},
    text={twiddling},
    description={Les opérations de twiddling sont les opérations de réarrangement des données de la MDCT effectuées autour de l'appel à la FFT}
}

\newglossaryentry{floating-point}{
    name={Floating point},
    text={floating point},
    description={Méthode d'encodage de nombres réels utilisée par les ordinateurs, équivalente à la notation scientifique en numération binaire}
}

\newglossaryentry{fixed-point}{
    name={Fixed point},
    text={fixed point},
    description={Méthode d'encodage de nombres possédant un nombre fixe de chiffres avant et après la virgule}
}

\newglossaryentry{window-function}{
    name={Window function},
    text={window function},
    description={Fonction appliquée à une fenêtre d'un signal temporel avant son analyse fréquentielle dans le but d'améliorer cette analyse}
}

\newglossaryentry{headroom}{
    name={Headroom},
    text={headroom},
    description={La headroom est la capacité d'un système de traitement de signal à dépasser ses limites sans être endommagé. Ce terme est utilisé ici pour désigner la marge de valeurs inutilisées à l'encodage du signal}
}

\newglossaryentry{overlap}{
    name={Overlap},
    text={overlap},
    description={Chevauchement, la MDCT utilise des fenêtres d'échantillons avec overlapping : la seconde moitié de la fenêtre analysée correspond à la première moitié de la fenêtre suivante}
}

\newglossaryentry{broadcast}{
    name={Broadcast},
    text={broadcast},
    description={Diffusion, utilisé dans ce travail pour désigner le domaine de la diffusion audiovisuelle}
}

\glsaddallunused
