\documentclass{article}

\usepackage[francais]{babel}
\usepackage[T1]{fontenc}
\usepackage{moreverb}       % verbatim with tab

\usepackage{wrapfig}
\usepackage{graphicx}
\usepackage{geometry}
\geometry{hmargin=2.5cm}
\usepackage{amsmath}
\usepackage{siunitx}

\usepackage{graphicx}
\usepackage{subcaption}
\usepackage{float}
\usepackage{hyperref}
\usepackage{setspace}
\usepackage{xcolor}
\usepackage{pdfpages}
\usepackage{enumitem}
\usepackage{lscape}

% https://tug.org/FontCatalogue/libertinusserif/
\usepackage{libertinus}
\usepackage[T1]{fontenc}

\usepackage{fancyhdr}       % en-têtes
\usepackage{lastpage}       % numéro de dernière page

\title{}

\title{Développement d'une solution de software embarqué sur processeur ARM pour encodage audio AAC optimisé aux applications d'EVS}
\date{2020 -- 2021}
\author{Laura Binacchi}

\pagestyle{fancy}
\renewcommand\headrulewidth{1pt}
\fancyhead[L]{Laura Binacchi}
\fancyhead[C]{Programmation procédurale}
\fancyhead[R]{\today}


\begin{document}
    \pagenumbering{gobble}
    \includepdf[pages={1}]{pdg}
    \newpage
    \tableofcontents
    \newpage
    \pagenumbering{arabic}

    \section*{Remerciements}
    \paragraph{}

    \section*{Introduction}
    \paragraph{}
    Développement d'une solution de software embarqué sur processeur ARM pour encodage audio AAC optimisé aux applications d'EVS :
    «\begin{itemize}
        \item Prise de connaissance de l'encodage AAC et de l'environnement EVS qui utilise ce type de format ;
        \item Prise de connaissance des résultats des optimisations possibles du modèle psycho-acoustique développé par EVS ;
        \item Développement du code en C ou Assembler pour l'encodage AAC sur plateforme ARM ;
        \item Test du système et documentation de son implémentation.
    \end{itemize}







    \section*{Conclusion}
    \paragraph{}

\end{document}